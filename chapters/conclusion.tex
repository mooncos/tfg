\chapter{Conclusions}

The main objective of this Bachelor's Thesis was to develop a radar node designed to provide real-time wireless data transmission while being low-cost and low-power.

The measurement and analysis of the signals produced by the RF and radar control boards has been carried out. An MCU has been selected and integrated into the system. The signals generated by the baseband board were incompatible with the range of input of the MCU ADC, therefore an additional conditioning PCB has been developed with further compatibility requirements in mind. This conditioning PCB has been simulated and tested. A DSP pipeline has been implemented including discard of useless samples, conversion to a suitable format, FFT computation and an optional step of frequency bins summation. The performance of the DSP pipeline has been evaluated. A wireless link to transmit radar data has been implemented by leveraging the Bluetooth Low Energy protocol stack, as well as its performance tested. A user application has been developed to evaluate the system performance and test the overall system operation. The system is capable of providing stable real-time measurements for up to \SI{6}{\metre} at a bitrate nearing \SI{700}{\kilo\bit\per\second}. Additionally, the application has been designed to display and store the information transmitted by the developed radar system. Finally, a measurement of the gait of a person has been carried out to test integration of all the parts of the system.

The development and evaluation of the new radar system in this project has been carried out in the context of gait analysis for Parkinson's disease diagnosis. Additionally, the system configuration has been deemed useful to the vital signals monitoring application. Notwithstanding its proven usefulness in the aforementioned applications, it is important to note that the developed system can be utilised in many radar application which requires real-time radar data transmission, processing or storage. Due to the characteristics of the new developed radar node, these applications will benefit from a compact, wireless, low-cost and low-power radar data solution that will enhance the capabilities of current radar sensing solutions.