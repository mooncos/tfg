% current font configuration
% \usepackage[default,scale=0.95]{opensans} % scaled-down version
%\usepackage[default]{sourcesanspro}
%\renewcommand\seriesdefault{l}
%\renewcommand\mddefault{l}
%\renewcommand\bfdefault{sb}% or \renewcommand\bfdefault{m}

\usepackage[utf8]{inputenc}
%\usepackage{lmodern}
%\usepackage{amsmath, amsfonts, amsthm, amssymb}  % Some math symbols
%\usepackage{amsmath, amsthm, amssymb}
%\usepackage{libertinus}
\usepackage{kpfonts}
\usepackage[T1]{fontenc}
\usepackage{textcomp}
%\usepackage[amsthm]{libertinust1math}
%\usepackage[scr=boondoxo,bb=boondox]{mathalpha}
\usepackage[british]{babel}
\usepackage[pdftex,dvipsnames]{xcolor}
\usepackage{mathtools}
\usepackage{hyperref}
\usepackage{xurl}
\usepackage{siunitx}
\usepackage{tikz}
\usepackage{csquotes}
\usepackage{amsmath}
\usepackage[capitalise]{cleveref}
\usepackage{longtable}
\usepackage{pdfpages}
\usepackage{titletoc}
\usepackage{graphicx}
\usepackage{scrextend}
\usepackage{subfig}
\usepackage{tabularray}
\usepackage{booktabs}
\usepackage{xargs}
\usepackage[colorinlistoftodos,prependcaption]{todonotes}
\usepackage[top=3cm,right=3cm,bottom=3cm,left=3cm]{geometry}
%\usepackage[numbers]{natbib}
%\usepackage[backend=biber,dateabbrev=false,urldate=iso,seconds=true,language=british,dashed=false,style=ieee]{biblatex}
\pdfminorversion=7

\hypersetup{
	hidelinks = true,
%	colorlinks = false,
%	linkcolor =  blue,
%	anchorcolor = blue,
%	urlcolor  = blue,
%	citecolor = blue,
}

\sisetup{
	separate-uncertainty=true,
	multi-part-units=single,
	detect-weight=true,
	per-mode=symbol,
}

\UseTblrLibrary{booktabs}
\UseTblrLibrary{siunitx}

% headings
\usepackage[
automark,
autooneside=false,% <- needed if you want to use \leftmark and \rightmark in a onesided document
headsepline
]{scrlayer-scrpage}
\pagestyle{scrheadings}
\clearpairofpagestyles
\ihead{\leftmark}
\ohead{\ifstr{\leftmark}{\rightbotmark}{}{\rightbotmark}}
\cfoot*{\pagemark}
\renewcommand*{\chaptermarkformat}{}
%\renewcommand*\pagestyle{scrheadings}% default pagestyle on chapter pages is plain

% annotations and todos
\newcommandx{\unsure}[2][1=]{\todo[linecolor=red,backgroundcolor=red!25,bordercolor=red,#1]{#2}}
\newcommandx{\change}[2][1=]{\todo[linecolor=blue,backgroundcolor=blue!25,bordercolor=blue,#1]{#2}}
\newcommandx{\info}[2][1=]{\todo[linecolor=OliveGreen,backgroundcolor=OliveGreen!25,bordercolor=OliveGreen,#1]{#2}}
\newcommandx{\improvement}[2][1=]{\todo[linecolor=Plum,backgroundcolor=Plum!25,bordercolor=Plum,#1]{#2}}
\newcommandx{\thiswillnotshow}[2][1=]{\todo[disable,#1]{#2}}

\graphicspath{{img/}{fig/}{fig/intro}}

\def \titulo{Design and characterisation of a continuous-wave linear frequency-modulated radar at 24 GHz for real-time applications}
\def \autor{Marcos Gómez Bracamonte}
\def \director{Ignacio Esteban López Delgado}
\def \ponente{Jesús Grajal de la Fuente}
\def \fecha{January 2023}
\def \fechad{Madrid}

\renewcommand\labelitemi{$-$}