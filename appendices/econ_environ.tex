\chapter{Ethical, economic, social and environmental aspects} \label{app:econ_environ}
\section{Introduction}
This Bachelor's Thesis belongs to the field of telecommunications and it has been developed at the ``Grupo de Microondas y Radar'' (GMR), a research group that belongs to the ``Señales, Sistemas y Radiocomunicaciones'' (SSR) department at the ``Escuela Técnica Superior de Ingenieros de Telecomunicación'' of the Universidad Politécnica de Madrid.

This project has developed and characterised a continuous-wave linear frequency modulated (CWLFM) radar for real-time applications, specifically for those needing live data collection and analysis, centring on the application of gait analysis for Parkinson's disease (PD) and vital signs monitoring.

\section{Description of relevant impacts related to the project}
\sectionmark{Description}
\subsection{Ethical and social aspects}

As mentioned through this text, the project has a direct relation with the medical sector, since it aims to provide an scalable solution that can aid health professionals in identifying and diagnosing PD or cardiorespiratory diseases in real-time. This holds a relevant ethical impact. Moreover, the device developed in this project allows to monitor and analyse this medical data from a distance and in a non-invasive way. In fact, there are situations where invasive methods cannot be used such as when monitoring infants or burn victims.

Additionally, the hardware developed in this project is highly replicable and easily manufactured. This provides a compact and scalable solution for use in small settings that allow the close monitoring of the elderly population. The scalability and low-power of the system can prove useful for use in developing areas with difficult access to high-end conventional medical diagnosing devices. Both reasons hold a significant social impact.

\subsection{Economic aspects}

One of the main advantages of the developed system is its low-cost. It reduces the cost of a radar node by more than 90\%, while being capable of providing the real-time useful data. It does so at a much reduced cost than current devices in use at hospitals. Moreover, the low cost enables the horizontal scalability for use with multi-static, multi-radar networks. It has been shown that current state-of-the-art solutions are significantly more expensive than the newly developed node and as such, providing good reliability and stability, it can have a significant economic impact.

\subsection{Environmental aspects}

The environmental impact is mainly involved in the manufacturing of the new radar node. The hardware design of the node has been carried out following the European Union's Conformity Standards \cite{EUConform2022} which limit the amount of harmful metals that electronic components have. In addition, the manufacture of the printed circuit boards (PCB) has been carried out with a lead-free process (RoHS) that substitutes the harmful lead metals with more sustainable and harmless alternatives such as tin-copper and gold platings. 

